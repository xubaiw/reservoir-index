\section{Formalization Structure}

This section gives a high-level overview of the definitions involved in the formalization of the Reactor model:

\begin{enumerate}
    \item Components of the Reactor model
    \item Means of changing and retrieving objects within a reactor tree
    \item Execution model
    \item Determinism and proof thereof
\end{enumerate}

Most code shown in this section is intended purely to aid explanation and will be covered in more detail in subsequent sections.

\subsection{Components of the Reactor Model}

Formalizing reactors will require us to define the following components.
The first two are rather a technicality, and can be glossed over.

\subsubsection{IDs}

Throughout the Reactor model, we use IDs to reference various kinds of components like ports, reactions, actions, etc.
The precise nature of IDs isn't relevant, and should remain ``opaque''. \footnote{The definition of reactors will implicitly impose some structure on IDs (the type needs to have at least as many distinct members as there are identifiable components in a reactor), but this does not affect their opaque nature.}
There are two ways of achieving this:

\begin{enumerate}
    \item We can make components like reactions and reactors generic over an ID-type (much like list- or set-types are generic over their element-type in many programming languages).
    \item We can define the type of IDs as a ``constant''.
        In Lean, constants can be considered as ``opaque'' definitions. 
        That is, the only thing we know about the object we're defining is its type -- we can never unfold its definition.
        Thus, by defining IDs as \lstinline{constant ID : Type}, we can use \lstinline{ID} throughout the model as if it were a generic type, without having to explicitly declare it as a type parameter on various components.
\end{enumerate}

We opt for the second approach as it reduces notation overhead and as we don't need the option of specifying the precise type of IDs in the model.

\subsubsection{Values}

Values are the objects upon which we perform computations in the Reactor model.
They are similar to IDs in that their precise structure is irrelevant -- we therefore define them as constants again.
One feature we \emph{do} need to impose on values is that they have a special element called the ``absent value'' (denoted by $\bot$).
This value is used to represent the absence of a value in ports.
We can still formalize this with constants, giving us an opaque type \lstinline{Value} and opaque element of this type called \lstinline{Value.absent}:

\begin{lstlisting}
constant _Value : Σ V : Type, V := Sigma.mk Unit Unit.unit
def Value : Type _ := _Value.fst
constant Value.absent : Value := _Value.snd

notation "⊥" => Value.absent
\end{lstlisting}

\subsubsection{Ports}

Ports are the interface points of a reactor, through which they can exchange values.
There are two kinds of ports: input and output ports.
The resulting definition reflects precisely these two aspects:

\begin{lstlisting}
inductive Port.Kind
  | «in» 
  | out

structure Port where
  kind : Port.Kind
  val : Value
\end{lstlisting}

\subsubsection{Changes \& Reactions}

Reactions are the basic computational units in the Reactor model. 
They take input values and output a set of ``changes'' to be realized in their reactor.

Changes are analogous to API-calls that can be performed by reactions in a Lingua Franca program.
The following can be read as defining the \lstinline{Change} type as an enumeration where each case has a payload:

\begin{lstlisting}
inductive Change
  | port (target : ID) (value : Value)
  | state (target : ID) (value : Value)
  | action (target : ID) (time : Time) (value : Value)
  | connect (src : ID) (dst : ID)
  | disconnect (src : ID) (dst : ID)
  | create (cl : Reactor.Class)
  | delete (rtr : ID)
\end{lstlisting}

If a reaction outputs a \lstinline{Change.port p v} this is analogous to a reaction calling \verb|SET| with port \verb|p| and value \verb|v| in Lingua Franca.
As a reaction can perform multiple API-calls in a single execution of its body, we formalize a reaction's body as producing a \emph{list} of changes:

\begin{lstlisting}
structure Reaction where
  body :        Input → List Change
  deps :        Port.Kind → Finset ID
  triggers :    Finset ID
  prio :        Priority
  tsSubInDeps : triggers ⊆ deps «.in»
  ... -- Additional constraints omitted.
\end{lstlisting}

Here we can also see that reactions have anti-/dependencies (which we also refer to as input and output dependencies), triggers and a priority.
The last field \lstinline{tsSubInDeps} shows a feature of dependently types languages like Lean: we can constrain instances of types to fulfill given propositions.
Here we place the constraint that for any instance of \lstinline{Reaction} it must hold that the set of triggers is a subset of the reaction's input dependencies.

\paragraph{Mutations}

We introduce a distinction between different types of reactions, based on whether they can produce ``mutating'' changes.
A change is considered mutating if it can change the \emph{structure} of a reactor.
This is the case for \lstinline{connect}, \lstinline{disconnect}, \lstinline{create} and \lstinline{delete}.

If there is any input for which a reaction produces a mutating change, we call that reaction a ``mutation''.
If a reaction never produces mutating changes, we call it a ``normal'' reaction.

\paragraph{Pure Reactions}

Another aspect by which we differentiate reactions is whether they are ``pure''.
A reaction is considered pure if its output does not depend on its reactor's state, and it only produces changes to ports and actions.

The purity of a reaction will become relevant when formalizing connections between ports through a special kind of reaction called a ``relay reaction'', which needs to have a specific priority that is only available to pure reactions. 

\subsubsection{Reactors}

Reactors are the building blocks which combine the previously described components:

\begin{lstlisting}
inductive Raw.Reactor 
  | mk 
    (ports : ID ⇉ Port)
    (state : ID ⇉ Value)
    (rcns :  ID ⇉ Reaction)
    (nest :  ID → Option Raw.Reactor)
    (acts :  ID ⇉ Time.Tag ⇉ Value)
\end{lstlisting}

The double-arrow symbol \lstinline{⇉} denotes a finite map (basically a hashmap or partial function defined on finitely many inputs).
Thus, a reactor consists of:

\begin{itemize}
    \item a set of identified ports
    \item a set of state variables, which are just identified values
    \item a set of identified reactions -- as previously mentioned, we will use a special kind of reaction to model connections between reactors' ports
    \item a set of identified nested reactors (we can't define these as a finite map yet, due to technical reasons detailed below)
    \item a set of identified actions, where each action is a mapping from a time tag to a value
\end{itemize}

Structuring the components of a reactor such that they are all ``hidden'' behind IDs will later allow us to easily change and retrieve components in a reactor tree by referring to their IDs.

\paragraph{Raw to Proper}

The type defined above is called \lstinline{Raw.Reactor} as it is missing some of the constraints that will need to be present in reactors.
For example, we need to constrain each input port to have at most one incoming connection.
For technical reasons (which relate to how Lean maps inductive definitions to its underlying mathematical theory), the recursive nature of reactors doesn't allow us to place constraints directly on the \lstinline{Raw.Reactor} type.
Instead, we will need to perform the following workaround:

\begin{enumerate}
    \item We define a ``raw'' reactor type without constraints (this is \lstinline{Raw.Reactor} as shown above).
    \item We define the required set of constraints using this raw reactor type. A raw reactor satisfying these constraints is then called ``directly well-formed''.
    \item We define a notion of (complete) ``well-formedness'', which holds if the raw reactor itself, as well as all of its nested reactors are directly well-formed.
    \item We define a ``proper'' \lstinline{Reactor} type as the type of well-formed raw reactors.
    \item We restate all of the constraints which were previously stated for raw reactors in terms of proper reactors and prove that the \lstinline{Reactor} type satisfies these constraints (we call this process of going from the raw world to the proper world ``lifting'').
\end{enumerate}

The crux of this process is to perform the lifting in such a way that proper reactors never leak their raw underpinnings.
Thus, any subsequent uses of reactors can completely ignore this workaround.

\subsection{Changing and Retrieving Objects}

As reactors form a rooted tree structure, we will often want to access components located at arbitrary points within this tree.
We could achieve this by traversing each reactor in the path from the root reactor to our target component and finally accessing the appropriate structure field on the immediate parent reactor. 
As this is a lot of overhead for what should be a simple access operation we will build an API that allows us to simply specify the ID and kind of the component we're trying to access.

\subsubsection{Unique IDs}

For this arbitrary access of objects to be possible, we need to require each object to have a unique ID (otherwise it wouldn't be clear which object we're trying to access).
As we will specify which \emph{kind} of component we want to access, this uniqueness only needs to hold within a component kind.

A component kind is given a simple enumeration:

\begin{lstlisting}
inductive Cmp
  | rtr -- Nested reactors
  | rcn -- Reactions
  | prt -- Ports
  | act -- Actions
  | stv -- State variables
\end{lstlisting}

We then formalize the notion of a path from a root reactor to a target component, which we call a \lstinline{Lineage}:

\begin{lstlisting}
inductive Lineage : Reactor → Cmp → ID → Type _
  | «end» cmp : (i ∈ (σ.cmp? cmp).ids) → Lineage σ cmp i
  | nest {cmp} : (Lineage rtr cmp i) → (σ.nest j = some rtr) → Lineage σ cmp i
\end{lstlisting}

ID-uniqueness is then achieved by requiring that lineages leading to a given component kind and ID are unique:

\begin{lstlisting}
theorem uniqueIDs (l₁ l₂ : Lineage σ cmp i) : l₁ = l₂ := ...
\end{lstlisting}

\subsubsection{Retrieving Objects}

The retrieval of objects from a reactor tree is built upon the concept of lineages.
The central function in this context is \lstinline{con?}:

\begin{lstlisting}
noncomputable def con? (σ : Reactor) (cmp : Cmp) (i : ID) : Option (Identified Reactor) := 
  if h : Nonempty (Lineage σ cmp i) then h.some.container else none
\end{lstlisting}

Given a root reactor and a target object (specified by component and ID), the function returns the immediate parent reactor of the target object if possible -- that is, if the given ID actually specifies an object of the given component kind in the root reactor's tree.
Based on \lstinline{con?}, we build the \lstinline{obj?} function, which returns the target object itself, if possible.

\subsubsection{Changing Objects}

Changing objects in a reactor tree has the same problem as retrieving objects, in that we need to build an API for conveniently accessing arbitrary objects in the reactor tree.

There's a much more important aspect we need to consider though in how we formalize changes on a reactor:
Any time we make a change we need to ensure that the reactor constraints aren't violated.
Note that this isn't a secondary issue we can deal with further down the road -- Lean won't allow us to instantiate a reactor object if we don't prove that it's constraints hold.

As not every change to a reactor is possible (e.g. we can't add a dependency to a reaction that is not an input port of its parent reactor), this isn't simply an annoyance in the implementation of our change-API, but literally makes a generic change-API impossible.
There are two potential solutions to this:

\begin{enumerate}
  \item Each call to the change-API needs to be accompanied by a proof term that allows us to prove that the change preserves reactor constraints.
  \item We formalize changes as a relation ...
\end{enumerate}
