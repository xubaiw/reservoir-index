% define first the square mod 3 theorem
Theorem Square mod 3: If $q$ is a natural not divisible by $3$, then $q^2 \mod 3 = 1$.

Considering the different possibilities of the modulo operation, we have $q \mod 3 \in \{ 0, 1, 2 \}$.

For the first case, as $q$ is not divisible by $3$, we have a contradiction.

For the second case, we have $q \mod 3 = 1$.

By definition of the modulo operation, we have $k \in \mathbb{N}$ such that $q = 3k + 1$.

By expansion, we have $q^2 = (3k+1)^2 = 9k^2+6k+1 = 3(3k^2 + 2k) + 1$.

Hence, $q^2 \mod 3 = 1$.

For the last case, by definition of the modulo operation, we have $k \in \mathbb{N}$ such that $q = 3k + 2$.

Following the same expansion, we have $q^2 = (3k+2)^2 = 3(3k^2+4k+1) +1$.

Hence, $q^2 \mod 3 = 1$ again.

% then define the next theorem (that uses the previous one)
Theorem square of q divisible by 3 means q is divisible by 3: If $q \in \mathbb{N}$ and $q^2$ is divisible by $3$, then $q$ is also divisible by $3$.

We will prove the contrapositive

By the "square mod 3" theorem, we have $q^2 \mod 3 = 1$

Hence $q^2$ is not divisible by $3$.